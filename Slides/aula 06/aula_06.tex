% Options for packages loaded elsewhere
\PassOptionsToPackage{unicode}{hyperref}
\PassOptionsToPackage{hyphens}{url}
%
\documentclass[
  9pt,
  ignorenonframetext,
]{beamer}
\usepackage{pgfpages}
\setbeamertemplate{caption}[numbered]
\setbeamertemplate{caption label separator}{: }
\setbeamercolor{caption name}{fg=normal text.fg}
\beamertemplatenavigationsymbolsempty
% Prevent slide breaks in the middle of a paragraph
\widowpenalties 1 10000
\raggedbottom
\setbeamertemplate{part page}{
  \centering
  \begin{beamercolorbox}[sep=16pt,center]{part title}
    \usebeamerfont{part title}\insertpart\par
  \end{beamercolorbox}
}
\setbeamertemplate{section page}{
  \centering
  \begin{beamercolorbox}[sep=12pt,center]{part title}
    \usebeamerfont{section title}\insertsection\par
  \end{beamercolorbox}
}
\setbeamertemplate{subsection page}{
  \centering
  \begin{beamercolorbox}[sep=8pt,center]{part title}
    \usebeamerfont{subsection title}\insertsubsection\par
  \end{beamercolorbox}
}
\AtBeginPart{
  \frame{\partpage}
}
\AtBeginSection{
  \ifbibliography
  \else
    \frame{\sectionpage}
  \fi
}
\AtBeginSubsection{
  \frame{\subsectionpage}
}
\usepackage{lmodern}
\usepackage{amssymb,amsmath}
\usepackage{ifxetex,ifluatex}
\ifnum 0\ifxetex 1\fi\ifluatex 1\fi=0 % if pdftex
  \usepackage[T1]{fontenc}
  \usepackage[utf8]{inputenc}
  \usepackage{textcomp} % provide euro and other symbols
\else % if luatex or xetex
  \usepackage{unicode-math}
  \defaultfontfeatures{Scale=MatchLowercase}
  \defaultfontfeatures[\rmfamily]{Ligatures=TeX,Scale=1}
\fi
\usetheme[]{Berkeley}
\usecolortheme{dove}
\usefonttheme{structurebold}
% Use upquote if available, for straight quotes in verbatim environments
\IfFileExists{upquote.sty}{\usepackage{upquote}}{}
\IfFileExists{microtype.sty}{% use microtype if available
  \usepackage[]{microtype}
  \UseMicrotypeSet[protrusion]{basicmath} % disable protrusion for tt fonts
}{}
\makeatletter
\@ifundefined{KOMAClassName}{% if non-KOMA class
  \IfFileExists{parskip.sty}{%
    \usepackage{parskip}
  }{% else
    \setlength{\parindent}{0pt}
    \setlength{\parskip}{6pt plus 2pt minus 1pt}}
}{% if KOMA class
  \KOMAoptions{parskip=half}}
\makeatother
\usepackage{xcolor}
\IfFileExists{xurl.sty}{\usepackage{xurl}}{} % add URL line breaks if available
\IfFileExists{bookmark.sty}{\usepackage{bookmark}}{\usepackage{hyperref}}
\hypersetup{
  pdftitle={ggplot e extensões},
  pdfauthor={Frederico Bertholini},
  hidelinks,
  pdfcreator={LaTeX via pandoc}}
\urlstyle{same} % disable monospaced font for URLs
\newif\ifbibliography
\usepackage{color}
\usepackage{fancyvrb}
\newcommand{\VerbBar}{|}
\newcommand{\VERB}{\Verb[commandchars=\\\{\}]}
\DefineVerbatimEnvironment{Highlighting}{Verbatim}{commandchars=\\\{\}}
% Add ',fontsize=\small' for more characters per line
\usepackage{framed}
\definecolor{shadecolor}{RGB}{248,248,248}
\newenvironment{Shaded}{\begin{snugshade}}{\end{snugshade}}
\newcommand{\AlertTok}[1]{\textcolor[rgb]{0.94,0.16,0.16}{#1}}
\newcommand{\AnnotationTok}[1]{\textcolor[rgb]{0.56,0.35,0.01}{\textbf{\textit{#1}}}}
\newcommand{\AttributeTok}[1]{\textcolor[rgb]{0.77,0.63,0.00}{#1}}
\newcommand{\BaseNTok}[1]{\textcolor[rgb]{0.00,0.00,0.81}{#1}}
\newcommand{\BuiltInTok}[1]{#1}
\newcommand{\CharTok}[1]{\textcolor[rgb]{0.31,0.60,0.02}{#1}}
\newcommand{\CommentTok}[1]{\textcolor[rgb]{0.56,0.35,0.01}{\textit{#1}}}
\newcommand{\CommentVarTok}[1]{\textcolor[rgb]{0.56,0.35,0.01}{\textbf{\textit{#1}}}}
\newcommand{\ConstantTok}[1]{\textcolor[rgb]{0.00,0.00,0.00}{#1}}
\newcommand{\ControlFlowTok}[1]{\textcolor[rgb]{0.13,0.29,0.53}{\textbf{#1}}}
\newcommand{\DataTypeTok}[1]{\textcolor[rgb]{0.13,0.29,0.53}{#1}}
\newcommand{\DecValTok}[1]{\textcolor[rgb]{0.00,0.00,0.81}{#1}}
\newcommand{\DocumentationTok}[1]{\textcolor[rgb]{0.56,0.35,0.01}{\textbf{\textit{#1}}}}
\newcommand{\ErrorTok}[1]{\textcolor[rgb]{0.64,0.00,0.00}{\textbf{#1}}}
\newcommand{\ExtensionTok}[1]{#1}
\newcommand{\FloatTok}[1]{\textcolor[rgb]{0.00,0.00,0.81}{#1}}
\newcommand{\FunctionTok}[1]{\textcolor[rgb]{0.00,0.00,0.00}{#1}}
\newcommand{\ImportTok}[1]{#1}
\newcommand{\InformationTok}[1]{\textcolor[rgb]{0.56,0.35,0.01}{\textbf{\textit{#1}}}}
\newcommand{\KeywordTok}[1]{\textcolor[rgb]{0.13,0.29,0.53}{\textbf{#1}}}
\newcommand{\NormalTok}[1]{#1}
\newcommand{\OperatorTok}[1]{\textcolor[rgb]{0.81,0.36,0.00}{\textbf{#1}}}
\newcommand{\OtherTok}[1]{\textcolor[rgb]{0.56,0.35,0.01}{#1}}
\newcommand{\PreprocessorTok}[1]{\textcolor[rgb]{0.56,0.35,0.01}{\textit{#1}}}
\newcommand{\RegionMarkerTok}[1]{#1}
\newcommand{\SpecialCharTok}[1]{\textcolor[rgb]{0.00,0.00,0.00}{#1}}
\newcommand{\SpecialStringTok}[1]{\textcolor[rgb]{0.31,0.60,0.02}{#1}}
\newcommand{\StringTok}[1]{\textcolor[rgb]{0.31,0.60,0.02}{#1}}
\newcommand{\VariableTok}[1]{\textcolor[rgb]{0.00,0.00,0.00}{#1}}
\newcommand{\VerbatimStringTok}[1]{\textcolor[rgb]{0.31,0.60,0.02}{#1}}
\newcommand{\WarningTok}[1]{\textcolor[rgb]{0.56,0.35,0.01}{\textbf{\textit{#1}}}}
\usepackage{graphicx}
\makeatletter
\def\maxwidth{\ifdim\Gin@nat@width>\linewidth\linewidth\else\Gin@nat@width\fi}
\def\maxheight{\ifdim\Gin@nat@height>\textheight\textheight\else\Gin@nat@height\fi}
\makeatother
% Scale images if necessary, so that they will not overflow the page
% margins by default, and it is still possible to overwrite the defaults
% using explicit options in \includegraphics[width, height, ...]{}
\setkeys{Gin}{width=\maxwidth,height=\maxheight,keepaspectratio}
% Set default figure placement to htbp
\makeatletter
\def\fps@figure{htbp}
\makeatother
\setlength{\emergencystretch}{3em} % prevent overfull lines
\providecommand{\tightlist}{%
  \setlength{\itemsep}{0pt}\setlength{\parskip}{0pt}}
\setcounter{secnumdepth}{5}

\title{ggplot e extensões}
\subtitle{Métodos Quantitativos Aplicados à Ciência Política}
\author{Frederico Bertholini}
\date{26.out.2020}

\begin{document}
\frame{\titlepage}

\begin{frame}[allowframebreaks]
  \tableofcontents[hideallsubsections]
\end{frame}
\begin{frame}[fragile]{Rode seus pacotes!}
\protect\hypertarget{rode-seus-pacotes}{}
\begin{Shaded}
\begin{Highlighting}[]
\KeywordTok{lapply}\NormalTok{(}\KeywordTok{c}\NormalTok{(}\StringTok{"tidyverse"}\NormalTok{,}\StringTok{"haven"}\NormalTok{,}\StringTok{"lubridate"}\NormalTok{,}
         \StringTok{"janitor"}\NormalTok{,}\StringTok{"readxl"}\NormalTok{,}
          \StringTok{"stringr"}\NormalTok{, }\StringTok{"magrittr"}\NormalTok{,}\StringTok{"srvyr"}\NormalTok{,}
         \StringTok{"survey"}\NormalTok{),require,}\DataTypeTok{character.only=}\NormalTok{T)}
\end{Highlighting}
\end{Shaded}
\end{frame}

\hypertarget{visualizauxe7uxe3o-de-dados}{%
\section{Visualização de dados}\label{visualizauxe7uxe3o-de-dados}}

\begin{frame}{onde estamos?}
\protect\hypertarget{onde-estamos}{}
\includegraphics{imgs/data-science-communicate.png}
\end{frame}

\begin{frame}{Uma exibição gráfica deve (1/2)}
\protect\hypertarget{uma-exibiuxe7uxe3o-gruxe1fica-deve-12}{}
Mostrar os dados

Induzir o observador a pensar em sua substância, não em metodologia ou
tecnologia de produção

Evitar distorcer o que os dados dizem

Apresentar muitos números em pequenos espaços

Tornar grandes conjuntos de dados coerentes
\end{frame}

\begin{frame}{Uma exibição gráfica deve (2/2)}
\protect\hypertarget{uma-exibiuxe7uxe3o-gruxe1fica-deve-22}{}
Encorajar o observador a comparar diferentes partes dos dados

Revelar diferentes níveis de detalhamento dos dados

Servir a um propósito claro e razoável: descrição, exploração, tabulação
ou decoração

Estar integrada com as descrições estatísticas e verbais do conjunto de
dados
\end{frame}

\begin{frame}{4 princípios (Edward Tufte)}
\protect\hypertarget{princuxedpios-edward-tufte}{}
Miniaturas Múltiplas

Menor diferença efetiva

Causalidade (Respondendo a pergunta: ``Comparado com o quê?'')

Contexto
\end{frame}

\begin{frame}{}
\protect\hypertarget{section}{}
\includegraphics{imgs/ideas_ink_space_time.png}
\end{frame}

\begin{frame}{O que você quer mostrar?}
\protect\hypertarget{o-que-vocuxea-quer-mostrar}{}
\begin{figure}
\centering
\includegraphics{imgs/Pic_2.png}
\caption{Andrew Abela Chart chooser}
\end{figure}
\end{frame}

\begin{frame}{}
\protect\hypertarget{section-1}{}
\href{https://www.techprevue.com/decision-tree-perfect-visualisation-data/}{\includegraphics{imgs/abela-chart-chooser.jpg}}

\href{https://www.youtube.com/watch?v=00zjDdXUcy4}{Animado}
\end{frame}

\begin{frame}{Princípios}
\protect\hypertarget{princuxedpios}{}
\begin{itemize}
\item
  O que você quer mostrar?
\item
  Elementos que podem \textbf{destacar} ou \textbf{confundir} o que você
  quer mostrar.
\item
  vamos tentar alternar ``teoria'' com live code
\item
  Ah, mas eu posso usar base R? Poder, pode\ldots{}
\end{itemize}
\end{frame}

\begin{frame}[fragile]{}
\protect\hypertarget{section-2}{}
\begin{Shaded}
\begin{Highlighting}[]
\KeywordTok{plot}\NormalTok{(mtcars}\OperatorTok{$}\NormalTok{wt, mtcars}\OperatorTok{$}\NormalTok{mpg)}
\end{Highlighting}
\end{Shaded}

\includegraphics{aula_06_files/figure-beamer/unnamed-chunk-2-1.pdf}
\end{frame}

\begin{frame}[fragile]{}
\protect\hypertarget{section-3}{}
\begin{Shaded}
\begin{Highlighting}[]
\KeywordTok{ggplot}\NormalTok{(mtcars, }\KeywordTok{aes}\NormalTok{(}\DataTypeTok{x =}\NormalTok{ wt, }\DataTypeTok{y =}\NormalTok{ mpg)) }\OperatorTok{+}
\StringTok{  }\KeywordTok{geom\_point}\NormalTok{()}
\end{Highlighting}
\end{Shaded}

\includegraphics{aula_06_files/figure-beamer/unnamed-chunk-3-1.pdf}
\end{frame}

\hypertarget{ggplot}{%
\section{ggplot}\label{ggplot}}

\begin{frame}{Recursos}
\protect\hypertarget{recursos}{}
\begin{itemize}
\item
  \href{https://r-graphics.org/}{R Graphics Cookbook}
\item
  \href{https://www.r-graph-gallery.com/}{R Graph Gallery}
\item
  \href{http://www.sthda.com/english/wiki/be-awesome-in-ggplot2-a-practical-guide-to-be-highly-effective-r-software-and-data-visualization}{STHDA}
\item
  \href{https://clauswilke.com/dataviz/}{Fundamentals of Data
  Visualization}
\item
  \href{http://r-statistics.co/ggplot2-cheatsheet.html}{r-statistics}
\item
  \href{https://exts.ggplot2.tidyverse.org/gallery/}{Extensões}
\end{itemize}
\end{frame}

\begin{frame}{Elementos do ggplot}
\protect\hypertarget{elementos-do-ggplot}{}
\begin{itemize}
\item
  Dados
\item
  Geometrias
\item
  Estéticas
\item
  Escalas (estética)
\item
  Escalas (eixos)
\item
  Tema
\item
  Facet
\end{itemize}
\end{frame}

\begin{frame}{Dados \texttt{data\ =}}
\protect\hypertarget{dados-data}{}
\begin{itemize}
\item
  Dado empilhado?
\item
  Cada coluna será uma entrada!
\end{itemize}
\end{frame}

\begin{frame}[fragile]{Geometrias \texttt{geom\_}}
\protect\hypertarget{geometrias-geom_}{}
\begin{itemize}
\item
  geom\_\texttt{tipo\_de\_geometria}
\item
  Recursos +
\item
  \href{https://www.rstudio.com/wp-content/uploads/2016/03/ggplot2-cheatsheet-portuguese.pdf}{cheat
  sheet}
\item
  \href{https://ggplot2.tidyverse.org/reference/}{manual ggplot}
\end{itemize}
\end{frame}

\begin{frame}{}
\protect\hypertarget{section-4}{}
\includegraphics{imgs/datavizCS.pdf}
\end{frame}

\begin{frame}{geometrias}
\protect\hypertarget{geometrias}{}
\url{https://ggplot2.tidyverse.org/reference/index.html\#section-layers}
\end{frame}

\begin{frame}[fragile]{Estéticas \texttt{aes()}}
\protect\hypertarget{estuxe9ticas-aes}{}
\begin{itemize}
\item
  \texttt{x} (\texttt{xmax} e \texttt{xmin})
\item
  \texttt{y}(\texttt{ymax} e \texttt{ymin})
\item
  \texttt{color}
\item
  \texttt{fill}
\item
  \texttt{shape}
\item
  \texttt{group}
\item
  \texttt{size}
\end{itemize}
\end{frame}

\begin{frame}[fragile]{Escalas (estética) \texttt{scale\_}}
\protect\hypertarget{escalas-estuxe9tica-scale_}{}
\begin{itemize}
\item
  \texttt{scale\_color\_xx}
\item
  \texttt{scale\_fill\_xx}
\item
  \texttt{scale\_shape\_xx}
\end{itemize}
\end{frame}

\begin{frame}[fragile]{Escalas (eixos) \texttt{scale\_x}}
\protect\hypertarget{escalas-eixos-scale_x}{}
\begin{itemize}
\item
  Contínua \texttt{scale\_x\_continuous}
\item
  Discreta \texttt{scale\_x\_discrete}
\item
  Tempo \texttt{scale\_yearmon}
\item
  Série de tempo \texttt{zoo}e \texttt{lubridate} --\textgreater{}
  \texttt{scale\_yearmon}
\end{itemize}
\end{frame}

\begin{frame}[fragile]{Tema}
\protect\hypertarget{tema}{}
\begin{itemize}
\item
  Customização total da visualização
\item
  Eixos
\item
  Texto \texttt{element\_text}
\item
  linhas de grade
\end{itemize}
\end{frame}

\begin{frame}{Facet}
\protect\hypertarget{facet}{}
\begin{itemize}
\item
  facet\_grid
\item
  facet\_wrap
\end{itemize}
\end{frame}

\begin{frame}[fragile]{Adicionais}
\protect\hypertarget{adicionais}{}
Gráficos com interatividade:

\begin{itemize}
\item
  \href{https://davidgohel.github.io/ggiraph/articles/offcran/using_ggiraph.html}{ggiraph}
\item
  \href{https://plotly-r.com/index.html}{plotly (\texttt{ggplotly})}
\end{itemize}

Combinação de gráficos

\begin{itemize}
\item
  \href{https://patchwork.data-imaginist.com/articles/patchwork.html}{patchwork}
\item
  \href{https://wilkelab.org/cowplot/articles/introduction.html}{cowplot}
\end{itemize}
\end{frame}

\begin{frame}[fragile]{Exercício}
\protect\hypertarget{exercuxedcio}{}
\begin{itemize}
\item
  Carregue os dados de exemplo do pacote survey \texttt{data(api)}, use
  o data.frame \texttt{apisrs}
\item
  Crie o objeto \texttt{tbl\_svy} com o nome \texttt{amostra\_expandida}
  expandindo a amostra aleatória simples usando apenas a variável
  (coluna) ``pw'', contendo o peso amostral. Dica: execute
  \texttt{as\_survey(weight=pw)}.
\item
  Usando a variável \texttt{stype} crie uma nova variável indicando se a
  escola é de nível fundamental (categorias \textbf{E} e \textbf{M} de
  \texttt{stype}) ou de nível médio (categoria \emph{H} de
  \texttt{stype}). Dica: use \texttt{mutate}e \texttt{case\_when}.
\item
  Faça um gráfico de barras comparando a variação média das notas de
  1999 (\texttt{api99}) e 2000 (\texttt{api00}) por nível e utilize as
  estimativas intervalares. Dica: olhe o código da aula 07, utilize
  \texttt{geom\_errorbar} para a estimativa intervalar.
\end{itemize}
\end{frame}

\begin{frame}[fragile]{Resolução}
\protect\hypertarget{resoluuxe7uxe3o}{}
\begin{Shaded}
\begin{Highlighting}[]
\KeywordTok{data}\NormalTok{(api)}

\NormalTok{amostra\_expandida \textless{}{-}}\StringTok{ }\NormalTok{apisrs }\OperatorTok{\%\textgreater{}\%}\StringTok{ }
\StringTok{  }\KeywordTok{as\_survey}\NormalTok{(}\DataTypeTok{weight =}\NormalTok{ pw) }\OperatorTok{\%\textgreater{}\%}
\StringTok{  }\KeywordTok{mutate}\NormalTok{(}\DataTypeTok{nivel=}\KeywordTok{case\_when}\NormalTok{(}
\NormalTok{    stype}\OperatorTok{==}\StringTok{"E"}\OperatorTok{\textasciitilde{}}\StringTok{"Fundamental"}\NormalTok{,}
\NormalTok{    stype}\OperatorTok{==}\StringTok{"M"}\OperatorTok{\textasciitilde{}}\StringTok{"Fundamental"}\NormalTok{,}
\NormalTok{    stype}\OperatorTok{==}\StringTok{"H"}\OperatorTok{\textasciitilde{}}\StringTok{"Médio"}
\NormalTok{  ))}
\end{Highlighting}
\end{Shaded}
\end{frame}

\begin{frame}[fragile]{}
\protect\hypertarget{section-5}{}
\begin{Shaded}
\begin{Highlighting}[]
\NormalTok{out \textless{}{-}}\StringTok{ }\NormalTok{amostra\_expandida }\OperatorTok{\%\textgreater{}\%}
\StringTok{  }\KeywordTok{group\_by}\NormalTok{(nivel) }\OperatorTok{\%\textgreater{}\%}
\StringTok{  }\KeywordTok{summarise}\NormalTok{(}\DataTypeTok{api\_diff =} 
              \KeywordTok{survey\_mean}\NormalTok{(api00 }\OperatorTok{{-}}\StringTok{ }\NormalTok{api99, }\DataTypeTok{vartype =} \StringTok{"ci"}\NormalTok{))}
\end{Highlighting}
\end{Shaded}
\end{frame}

\begin{frame}[fragile]{}
\protect\hypertarget{section-6}{}
\begin{Shaded}
\begin{Highlighting}[]
\NormalTok{grafico \textless{}{-}}\StringTok{ }\NormalTok{out }\OperatorTok{\%\textgreater{}\%}\StringTok{ }
\StringTok{  }\KeywordTok{ggplot}\NormalTok{(}\KeywordTok{aes}\NormalTok{(}\DataTypeTok{x =}\NormalTok{ nivel, }\DataTypeTok{y =}\NormalTok{ api\_diff, }
             \DataTypeTok{fill =}\NormalTok{ nivel,}\DataTypeTok{color=}\NormalTok{nivel,}
                       \DataTypeTok{ymax =}\NormalTok{ api\_diff\_upp, }
             \DataTypeTok{ymin =}\NormalTok{ api\_diff\_low)) }\OperatorTok{+}
\StringTok{  }\KeywordTok{geom\_bar}\NormalTok{(}\DataTypeTok{stat =} \StringTok{"identity"}\NormalTok{,}\DataTypeTok{alpha=}\FloatTok{0.6}\NormalTok{) }\OperatorTok{+}
\StringTok{  }\KeywordTok{geom\_errorbar}\NormalTok{(}\DataTypeTok{width =} \DecValTok{0}\NormalTok{,}\DataTypeTok{size=}\DecValTok{3}\NormalTok{) }
\end{Highlighting}
\end{Shaded}
\end{frame}

\begin{frame}[fragile]{}
\protect\hypertarget{section-7}{}
\begin{Shaded}
\begin{Highlighting}[]
\NormalTok{grafico }
\end{Highlighting}
\end{Shaded}

\includegraphics{aula_06_files/figure-beamer/unnamed-chunk-7-1.pdf}
\end{frame}

\begin{frame}[fragile]{}
\protect\hypertarget{section-8}{}
\begin{Shaded}
\begin{Highlighting}[]
\NormalTok{grafico }\OperatorTok{+}\StringTok{ }\KeywordTok{labs}\NormalTok{(}\DataTypeTok{y=}\StringTok{"Variação das notas"}\NormalTok{,}\DataTypeTok{x=}\StringTok{""}\NormalTok{,}\DataTypeTok{color=}\StringTok{"Nível"}\NormalTok{,}\DataTypeTok{fill=}\StringTok{"Nível"}\NormalTok{) }\OperatorTok{+}\StringTok{ }\KeywordTok{theme\_minimal}\NormalTok{()}
\end{Highlighting}
\end{Shaded}

\includegraphics{aula_06_files/figure-beamer/unnamed-chunk-8-1.pdf}
\end{frame}

\end{document}
